% interactcadsample.tex
% v1.03 - April 2017

\documentclass[]{interact}

\usepackage{epstopdf}% To incorporate .eps illustrations using PDFLaTeX, etc.
\usepackage{subfigure}% Support for small, `sub' figures and tables
%\usepackage[nolists,tablesfirst]{endfloat}% To `separate' figures and tables from text if required

\usepackage{natbib}% Citation support using natbib.sty
\bibpunct[, ]{(}{)}{;}{a}{}{,}% Citation support using natbib.sty
\renewcommand\bibfont{\fontsize{10}{12}\selectfont}% Bibliography support using natbib.sty

\theoremstyle{plain}% Theorem-like structures provided by amsthm.sty
\newtheorem{theorem}{Theorem}[section]
\newtheorem{lemma}[theorem]{Lemma}
\newtheorem{corollary}[theorem]{Corollary}
\newtheorem{proposition}[theorem]{Proposition}

\theoremstyle{definition}
\newtheorem{definition}[theorem]{Definition}
\newtheorem{example}[theorem]{Example}

\theoremstyle{remark}
\newtheorem{remark}{Remark}
\newtheorem{notation}{Notation}


% tightlist command for lists without linebreak
\providecommand{\tightlist}{%
  \setlength{\itemsep}{0pt}\setlength{\parskip}{0pt}}



\usepackage{hyperref}
\usepackage[utf8]{inputenc}
\def\tightlist{}


\begin{document}


\articletype{ORIGINAL RESEARCH ARTICLE}

\title{Diagnostic biface acquisition through trade with distinct
extralocal communities of practice: A case study from the American
Southeast}


\author{\name{Robert Z. Selden, Jr.$^{a}$, Catherine G.
Cooper$^{b}$, John E. Dockall$^{c}$}
\affil{$^{a}$Heritage Research Center, Stephen F. Austin State
University; Department of Biology, Stephen F. Austin State University;
Texas Archeological Research Laboratory, The University of Texas at
Austin; and Cultural Heritage Department, Jean Monnet
University; $^{b}$National Center for Preservation Technology and
Training, National Park Service; $^{c}$Stantec, Inc.}
}

\thanks{CONTACT Robert Z. Selden,
Jr.. Email: \href{mailto:zselden@sfasu.edu}{\nolinkurl{zselden@sfasu.edu}}, Catherine
G. Cooper. Email: , John E. Dockall. Email: }

\maketitle

\begin{abstract}
Nodule size and mechanical flaking properties have been advanced to
account for assemblage level differences in the morphology of stone
tools within regions where raw material is abundant. However, in
instances where lithic raw materials---larger nodules in
particular---were scarce, did traders interact with brokers from
distinct extralocal communities to acquire large bifaces? Gahagan
bifaces are among those implements diagnostic of Formative/Early Caddo
material culture (CE 800 - 1250); although, due to a lack of local
production evidence, it is also widely accepted that Gahagan bifaces
were not manufactured by the Caddo. This study asks whether Caddo
traders engaged in commerce with brokers from discrete extralocal
communities of practice to acquire Gahagan bifaces. Results demonstrate
significant differences in Gahagan biface geochemistry and shape based
on color group assignment, suggesting that Gahagan biface shape was
conditioned by raw material color. Given these findings, it is posited
that Caddo traders acquired Gahagan bifaces from two distinct
communities of practice, where local raw material and production
differences resulted in Gahagan bifaces that were unique in color,
geochemistry, and shape.
\end{abstract}

\begin{keywords}
American Southeast; Caddo; NAGPRA; lithics; raw material color; pXRF;
computational archaeology; museum studies; digital humanities;
non-Western art history; STEM; STEAM
\end{keywords}

\begin{quote}
``The question of questions for mankind---the problem which underlies
all others, and is more deeply interesting than any other---is the
ascertainment of the place which man occupies in nature, and of his
relation to the universe of things.'' \textbf{--H. Thomas Henry Huxley},
\emph{Man's Place in Nature}
\end{quote}

\hypertarget{introduction}{%
\section{Introduction}\label{introduction}}

\hypertarget{methods-and-results}{%
\section{Methods and Results}\label{methods-and-results}}

\hypertarget{discussion-and-conclusion}{%
\section{Discussion and Conclusion}\label{discussion-and-conclusion}}

\hypertarget{acknowledgements}{%
\section*{Acknowledgement(s)}\label{acknowledgements}}
\addcontentsline{toc}{section}{Acknowledgement(s)}

We extend our gratitude to the Caddo Nation of Oklahoma, the Caddo
Nation Tribal Council, Tribal Chairman, and Tribal Historic Preservation
Office for their continued guidance and support of our work, as well as
access to NAGPRA and previously repatriated collections. Thanks also to
the Williamson Museum and the Louisiana State Exhibit Museum for
providing access to the Gahagan bifaces, and to Bruce Kayser for his
time and guidance with pXRF questions. RZS also extends his gratitude to
Emma Sherratt, Kersten Bergstrom, Lauren Butaric, Dean C. Adams, and
Michael L. Collyer for their constructive criticisms and suggestions
throughout the development of this research program.

\hypertarget{disclosure-statement}{%
\section*{Disclosure statement}\label{disclosure-statement}}
\addcontentsline{toc}{section}{Disclosure statement}

The authors declare no conflicts of interest.

\hypertarget{data-management}{%
\section*{Data management}\label{data-management}}
\addcontentsline{toc}{section}{Data management}

All data and analysis code associated with this project are openly
available through the
\href{https://github.com/seldenlab/gahaganmorph.5}{GitHub} repository,
which is digitally curated on the Open Science Framework
(\href{https://osf.io/3jb94/}{DOI 10.17605/OSF.IO/3JB94}). Additionally,
images of all Gahagan bifaces used in this study are openly available to
view/download through an open access comparative collection
(\url{https://scholarworks.sfasu.edu/ita-gahaganbiface/}). The
supplementary materials include all analysis data and code used in the
study, providing a means for others to reproduce (exactly) those results
discussed and expounded upon in this article. The replicable nature of
this undertaking provides others with the means to critically assess and
evaluate the various analytical components of this study
\citep{RN20915, RN20916, RN20917}, which is a necessary requirement for
the production of reliable knowledge.

Reproducibility projects in \href{https://osf.io/ezcuj/}{psychology} and
\href{https://www.cos.io/rpcb}{cancer biology} are impacting current
research practices across all domains. Examples of reproducible research
are becoming more abundant in archaeology
\citep{RN20804, RN21009, RN21001, RN9364, RN11097}, and the next
generation of archaeologists are learning those tools and methods needed
to reproduce and/or replicate research results \citep{RN21007}.
Reproducible and replicable research work flows are often employed at
the highest levels of humanities-based inquiries to mitigate concern or
doubt regarding proper execution, and is of particular import should the
results have---explicitly or implicitly---a major impact on scientific
progress \citep{RN21008}.

\hypertarget{funding}{%
\section*{Funding}\label{funding}}
\addcontentsline{toc}{section}{Funding}

Components of the analytical workflow were developed and funded by a
Preservation Technology and Training grant (P14AP00138) to RZS from the
National Center for Preservation Technology and Training, as well as
grants to RZS from the Caddo Nation of Oklahoma, National Forests and
Grasslands in Texas (15-PA-11081300-033) and the United States Forest
Service (20-PA-11081300-074). Additional funding and logistical support
was provided by the Heritage Research Center at Stephen F. Austin State
University and the National Center for Preservation Technology and
Training.

\bibliographystyle{tfcad}
\bibliography{article.bib}





\end{document}
